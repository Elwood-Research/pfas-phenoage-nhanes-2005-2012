\documentclass[10pt]{beamer}
\usetheme{Madrid}
\usecolortheme{default}
\usepackage{graphicx}
\usepackage{booktabs}
\usepackage{adjustbox}
\usepackage{array}

\title{PFAS and Biological Aging: NHANES 2005-2012 Analysis}
\subtitle{A Cross-Sectional Study of Per- and Polyfluoroalkyl Substances and PhenoAge}
\author{Elwood Research Team}
\date{February 13, 2026}

\begin{document}

\frame{\titlepage}

% Slide 2: Background
\begin{frame}{Background: PFAS Contamination}
\textbf{PFAS: Persistent Environmental Contaminants}
\begin{itemize}
\item \textbf{``Forever chemicals''}: Synthetic compounds resistant to degradation
\item \textbf{Ubiquitous exposure}: 98\% of U.S. population has detectable PFAS
\item \textbf{Known health effects}:
  \begin{itemize}
  \item Liver damage
  \item Immune dysfunction
  \item Metabolic disorders
  \item Cardiovascular disease
  \item Endocrine disruption
  \end{itemize}
\end{itemize}

\vspace{0.3cm}
\textbf{Study Rationale}: PFAS may accelerate biological aging through oxidative stress, inflammation, and metabolic dysregulation
\end{frame}

% Slide 3: PhenoAge Biomarker
\begin{frame}{PhenoAge: Validated Aging Biomarker}
\textbf{Developed by Levine et al. (2018)}

\vspace{0.3cm}
\textbf{Algorithm based on 9 biomarkers + chronological age}:
\begin{itemize}
\item Albumin, creatinine, glucose, C-reactive protein
\item Lymphocyte \%, mean corpuscular volume, red cell distribution width
\item Alkaline phosphatase, white blood cell count
\end{itemize}

\vspace{0.3cm}
\textbf{PhenoAge Acceleration = PhenoAge - Chronological Age}
\begin{itemize}
\item Positive values = accelerated aging
\item Negative values = decelerated aging
\end{itemize}

\vspace{0.3cm}
\textbf{Validated predictor}: All-cause mortality, CVD risk, cancer incidence
\end{frame}

% Slide 4: Research Question
\begin{frame}{Study Objectives}
\textbf{Primary Aim}

Examine associations between serum PFAS concentrations and PhenoAge acceleration in U.S. adults

\vspace{0.5cm}
\textbf{Research Questions}
\begin{enumerate}
\item Are PFAS concentrations associated with PhenoAge acceleration?
\item Do associations differ by PFAS compound?
\item Are associations independent of demographics and SES?
\item Do effects vary by sex or age group?
\end{enumerate}
\end{frame}

% Slide 5: Methods - Study Design
\begin{frame}{Methods: Study Design}
\textbf{NHANES 2005-2012}
\begin{itemize}
\item National Health and Nutrition Examination Survey
\item Cross-sectional, nationally representative
\item Cycles: 2005-2006, 2007-2008, 2009-2010, 2011-2012
\end{itemize}

\vspace{0.3cm}
\textbf{Inclusion Criteria}
\begin{itemize}
\item Adults aged $\geq$18 years
\item Non-pregnant
\item Complete PFAS measurements (all 4 compounds)
\item Complete PhenoAge biomarkers (all 9 components)
\item No extreme outliers ($|z|$ \textgreater{} 4)
\end{itemize}

\vspace{0.3cm}
\textbf{Final Sample}: N = 3,198 participants
\end{frame}

% Slide 6: PFAS Exposure
\begin{frame}{Methods: PFAS Exposure}
\textbf{Four Legacy PFAS Compounds}
\begin{enumerate}
\item \textbf{PFOA} (perfluorooctanoic acid)
\item \textbf{PFOS} (perfluorooctane sulfonic acid)
\item \textbf{PFHxS} (perfluorohexane sulfonic acid)
\item \textbf{PFNA} (perfluorononanoic acid)
\end{enumerate}

\vspace{0.3cm}
\textbf{Measurement}
\begin{itemize}
\item Serum concentrations (ng/mL)
\item CDC laboratory analysis
\item Natural log-transformed for regression
\end{itemize}
\end{frame}

% Slide 7: Statistical Analysis
\begin{frame}{Methods: Statistical Analysis}
\textbf{Progressive Regression Models}

\vspace{0.3cm}
\textbf{Model 1 (Crude)}: PFAS + age + sex

\vspace{0.2cm}
\textbf{Model 2 (Demographic-adjusted)}: Model 1 + race/ethnicity

\vspace{0.2cm}
\textbf{Model 3 (Fully-adjusted)}: Model 2 + education + poverty-income ratio

\vspace{0.5cm}
\textbf{Additional Analyses}
\begin{itemize}
\item Sex-stratified models
\item Age-stratified models ($<$50 vs. $\geq$50 years)
\item Sensitivity analyses
\item Weighted quantile sum (WQS) mixture regression
\end{itemize}
\end{frame}

% Slide 8: Sample Characteristics
\begin{frame}{Results: Sample Characteristics}
\textbf{Demographics}
\begin{itemize}
\item Mean age: 47.4 $\pm$ 19.2 years
\item Sex: 48.5\% male, 51.5\% female
\item Race: 72.1\% Non-Hispanic White, 10.8\% Non-Hispanic Black
\end{itemize}

\vspace{0.3cm}
\textbf{PFAS Exposure (Median, ng/mL)}
\begin{table}[h]
\centering
\small
\begin{tabular}{lll}
\toprule
\textbf{Compound} & \textbf{Median} & \textbf{IQR} \\
\midrule
PFOA & 3.30 & 2.20--4.88 \\
PFOS & 12.30 & 7.20--20.48 \\
PFHxS & 1.60 & 0.90--2.80 \\
PFNA & 1.10 & 0.80--1.64 \\
\bottomrule
\end{tabular}
\end{table}
\end{frame}

% Slide 9: Main Results
\begin{frame}{Results: Main Findings}
\textbf{Associations with PhenoAge Acceleration}

\textbf{Fully-Adjusted Model (Model 3)}

\vspace{0.3cm}
\begin{table}[h]
\centering
\small
\begin{tabular}{lccc}
\toprule
\textbf{Compound} & \textbf{$\beta$ (years)} & \textbf{95\% CI} & \textbf{p-value} \\
\midrule
PFOA & -1.90 & (-2.14, -1.67) & $<$0.001 \\
PFOS & -1.32 & (-1.52, -1.13) & $<$0.001 \\
PFHxS & -1.29 & (-1.47, -1.11) & $<$0.001 \\
PFNA & -1.26 & (-1.51, -1.02) & $<$0.001 \\
\bottomrule
\end{tabular}
\end{table}

\vspace{0.3cm}
\textbf{Interpretation}: All four PFAS compounds showed \textbf{significant INVERSE} associations with PhenoAge acceleration

(Higher PFAS $\rightarrow$ Lower biological aging)
\end{frame}

% Slide 10: Forest Plot
\begin{frame}{Results: Forest Plot}
\begin{center}
\includegraphics[width=0.85\textwidth]{../04-analysis/outputs/figures/figure4_forest_plot.png}
\end{center}
\textbf{Key Observation}: Consistent inverse associations across all compounds and models
\end{frame}

% Slide 11: STROBE Flow
\begin{frame}{Results: STROBE Flow Diagram}
\begin{center}
\includegraphics[width=0.75\textwidth]{../04-analysis/outputs/figures/figure1_strobe_flow.png}
\end{center}
\small
Initial sample: 9,226 $\rightarrow$ Final: 3,198 (35\% retention after exclusions)
\end{frame}

% Slide 12: Paradoxical Findings
\begin{frame}{Discussion: Paradoxical Findings}
\textbf{Unexpected Results}

\vspace{0.3cm}
\textbf{Hypothesis}: PFAS would \textit{accelerate} aging (positive associations)

\textbf{Findings}: All PFAS showed \textit{inverse} associations (negative)

\vspace{0.5cm}
\begin{alertblock}{Critical Interpretation}
\textbf{These results should NOT be interpreted as:}
\begin{itemize}
\item Evidence that PFAS is ``safe'' or ``protective''
\item Reason to reduce PFAS regulation
\end{itemize}
\end{alertblock}
\end{frame}

% Slide 13: Explanations
\begin{frame}{Discussion: Potential Explanations}
\textbf{1. Survival Bias (Most Likely)}
\begin{itemize}
\item Individuals most susceptible to PFAS may have died before study
\item Only healthiest, most resilient PFAS-exposed individuals remain
\item Stronger in older adults where survival bias operates more
\end{itemize}

\vspace{0.3cm}
\textbf{2. Reverse Causation}
\begin{itemize}
\item Biological aging may influence PFAS metabolism/excretion
\item Healthier individuals may retain PFAS longer
\item Cross-sectional design cannot establish temporal sequence
\end{itemize}

\vspace{0.3cm}
\textbf{3. Residual Confounding}
\begin{itemize}
\item Diet quality (seafood increases PFAS but provides nutrients)
\item Socioeconomic factors not fully captured
\item Occupational and geographic variation
\end{itemize}
\end{frame}

% Slide 14: Biological Context
\begin{frame}{Discussion: Biological Context}
\textbf{Established PFAS Toxicity Mechanisms}
\begin{itemize}
\item Oxidative stress
\item Chronic inflammation
\item Endocrine disruption
\item Mitochondrial dysfunction
\item Hepatotoxicity and nephrotoxicity
\item Immunotoxicity
\end{itemize}

\vspace{0.5cm}
\textbf{Expected}: These mechanisms should \textbf{accelerate} biological aging

\textbf{Observed}: Inverse associations in cross-sectional data

\vspace{0.3cm}
\textbf{Conclusion}: Cross-sectional design \textbf{cannot capture} causal processes linking PFAS to aging
\end{frame}

% Slide 15: Strengths
\begin{frame}{Study Strengths}
\textbf{Methodological Rigor}
\begin{itemize}
\item Large, nationally representative sample (N=3,198)
\item Validated biological aging biomarker (PhenoAge)
\item Standardized CDC laboratory measurements
\item Comprehensive covariate adjustment
\item Multiple PFAS compounds examined
\item Sex and age stratification
\item Sensitivity analyses conducted
\end{itemize}

\vspace{0.3cm}
\textbf{Novel Contributions}
\begin{itemize}
\item First comprehensive NHANES analysis of PFAS and PhenoAge
\item Rigorous statistical approach
\item Identification of paradoxical patterns
\end{itemize}
\end{frame}

% Slide 16: Limitations
\begin{frame}{Study Limitations}
\textbf{Critical Constraints}

\begin{itemize}
\item \textbf{Cross-sectional design}
  \begin{itemize}
  \item Cannot establish causality
  \item Vulnerable to reverse causation and survival bias
  \end{itemize}
\item \textbf{Single timepoint measurement}
  \begin{itemize}
  \item Does not capture lifetime exposure
  \item PFAS half-lives: 2-9 years
  \end{itemize}
\item \textbf{Survival bias} (cannot be fully addressed)
  \begin{itemize}
  \item Most affected individuals excluded by design
  \end{itemize}
\item \textbf{Unmeasured confounding}
  \begin{itemize}
  \item Diet, occupation, genetics not fully captured
  \end{itemize}
\end{itemize}
\end{frame}

% Slide 17: Public Health
\begin{frame}{Public Health Implications}
\begin{alertblock}{Critical Interpretation}
\textbf{DO NOT interpret as:}
\begin{itemize}
\item Evidence that PFAS is safe
\item Reason to halt regulation
\end{itemize}

\textbf{WHY?} Paradoxical findings likely due to methodological limitations, not true protective effects
\end{alertblock}

\vspace{0.3cm}
\textbf{Recommendations (Unchanged)}
\begin{itemize}
\item Continue PFAS exposure reduction
\item Maintain environmental regulations
\item Continue biomonitoring
\item Prioritize longitudinal research
\item Focus on vulnerable populations
\end{itemize}
\end{frame}

% Slide 18: Future Research
\begin{frame}{Future Research Priorities}
\textbf{1. Longitudinal Cohort Studies}
\begin{itemize}
\item Repeated PFAS and PhenoAge measurements
\item Establish temporal relationships
\item Minimize survival bias
\end{itemize}

\vspace{0.3cm}
\textbf{2. Mechanistic \& Multi-Omics Research}
\begin{itemize}
\item Epigenetic clocks, telomeres
\item Transcriptomic, proteomic, metabolomic aging markers
\end{itemize}

\vspace{0.3cm}
\textbf{3. Vulnerable Populations}
\begin{itemize}
\item Prenatal exposures, pregnancy, occupational cohorts
\end{itemize}

\vspace{0.3cm}
\textbf{4. Advanced Causal Inference}
\begin{itemize}
\item Mendelian randomization, target trial emulation
\end{itemize}
\end{frame}

% Slide 19: Conclusions
\begin{frame}{Conclusions}
\textbf{Key Takeaways}

\begin{enumerate}
\item \textbf{Paradoxical inverse associations} between PFAS and PhenoAge in cross-sectional NHANES data

\item \textbf{NOT evidence of safety}: Likely methodological limitations (survival bias, reverse causation, confounding)

\item \textbf{Contradicts toxicology}: Extensive evidence shows PFAS causes oxidative stress, inflammation, organ damage

\item \textbf{Public health stance unchanged}: Precautionary PFAS reduction remains essential

\item \textbf{Research imperative}: Longitudinal studies critically needed

\item \textbf{Methodological lesson}: Cross-sectional designs fundamentally limited for causal inference
\end{enumerate}
\end{frame}

% Slide 20: Final Thoughts
\begin{frame}{Final Thoughts}
\begin{block}{Association $\neq$ Causation}
Cross-sectional studies excel at hypothesis generation but are limited for causal inference
\end{block}

\vspace{0.3cm}
\textbf{This Study Demonstrates:}
\begin{itemize}
\item Importance of study design
\item Need for mechanistic understanding
\item Value of triangulation across study types
\item Critical thinking when findings contradict biology
\end{itemize}

\vspace{0.5cm}
\textbf{Moving Forward}

\textit{Rigorous longitudinal research + mechanistic studies + intervention trials = Better understanding of PFAS and aging}
\end{frame}

% Slide 21: Acknowledgments
\begin{frame}{Acknowledgments \& Contact}
\textbf{Data Source}
\begin{itemize}
\item NHANES 2005-2012 (CDC/NCHS)
\item CDC Division of Laboratory Sciences
\end{itemize}

\vspace{0.3cm}
\textbf{Transparency}
\begin{itemize}
\item Funding: None
\item Conflicts of Interest: None
\item Data: Publicly available through CDC
\end{itemize}

\vspace{0.5cm}
\textbf{Contact}

Elwood Research Team\\
elwoodresearch@gmail.com

\vspace{0.3cm}
\centering
\textbf{Questions?}
\end{frame}

\end{document}
