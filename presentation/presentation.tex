\documentclass[10pt]{beamer}
\usetheme{Madrid}
\usecolortheme{default}
\usepackage{graphicx}
\usepackage{booktabs}
\usepackage{adjustbox}

\title{Paradoxical Inverse Associations Between PFAS and Biological Aging}
\subtitle{A Cross-Sectional Analysis of NHANES 2005-2012}
\author{Elwood Research}
\date{February 13, 2026}

\begin{document}

\frame{\titlepage}

% Slide 2: Background
\begin{frame}{Background: PFAS and Health}
\begin{itemize}
    \item \textbf{PFAS}: Persistent environmental pollutants ("forever chemicals")
    \item Ubiquitous human exposure despite regulatory phase-outs
    \item Known toxicological effects:
    \begin{itemize}
        \item Hepatotoxicity, immunotoxicity
        \item Endocrine disruption
        \item Cardiometabolic dysfunction
    \end{itemize}
    \item \textbf{Key Question}: Do PFAS accelerate biological aging?
\end{itemize}
\end{frame}

% Slide 3: PhenoAge Biomarker
\begin{frame}{Biological Aging Measurement: PhenoAge}
\begin{itemize}
    \item \textbf{PhenoAge} (Levine et al. 2018): Validated composite biomarker
    \item Integrates chronological age + 9 blood chemistry markers:
    \begin{itemize}
        \item Albumin, creatinine, glucose, CRP
        \item Lymphocyte \%, MCV, RDW, alkaline phosphatase, WBC
    \end{itemize}
    \item \textbf{PhenoAge Acceleration}: PhenoAge - Chronological Age
    \begin{itemize}
        \item Positive = Faster biological aging
        \item Negative = Slower biological aging
    \end{itemize}
    \item Predicts mortality, morbidity, healthspan
\end{itemize}
\end{frame}

% Slide 4: Research Question
\begin{frame}{Research Question and Hypothesis}
\textbf{Research Question:}
\begin{quote}
Are serum PFAS concentrations associated with accelerated biological aging (PhenoAge) in U.S. adults?
\end{quote}

\vspace{0.5cm}

\textbf{Hypothesis:}
\begin{itemize}
    \item \textbf{Expected}: Higher PFAS $\rightarrow$ Accelerated aging (positive association)
    \item Based on:
    \begin{itemize}
        \item Toxicological evidence (oxidative stress, inflammation)
        \item Yan et al. 2025: Positive associations in UK Biobank (longitudinal)
    \end{itemize}
\end{itemize}
\end{frame}

% Slide 5: Methods
\begin{frame}{Methods: Study Design and Sample}
\textbf{Data Source:} NHANES 2005-2012

\vspace{0.3cm}

\textbf{Sample Selection:}
\begin{itemize}
    \item Adults $\geq$ 18 years
    \item Measured PFAS: PFOA, PFOS, PFHxS, PFNA
    \item Complete biomarker data for PhenoAge calculation
    \item \textbf{Final analytic sample: N = 3,198}
\end{itemize}

\vspace{0.3cm}

\textbf{Statistical Analysis:}
\begin{itemize}
    \item Survey-weighted linear regression
    \item Outcome: PhenoAge acceleration (years)
    \item Exposure: Log-transformed PFAS (ng/mL)
    \item 3 models: Crude, +demographics, +socioeconomic
\end{itemize}
\end{frame}

% Slide 6: Sample Characteristics
\begin{frame}{Sample Characteristics}
\begin{table}
\centering
\footnotesize
\begin{tabular}{lc}
\toprule
\textbf{Characteristic} & \textbf{Value} \\
\midrule
N & 3,198 \\
Age, mean (SD) & 47.4 (19.1) years \\
Female (\%) & 50.8\% \\
PhenoAge, mean (SD) & 45.6 (18.0) years \\
PhenoAge acceleration, mean (SD) & -1.80 (6.00) years \\
\midrule
\textbf{PFAS Concentrations (ng/mL), median (IQR):} & \\
PFOA & 3.30 (2.20-4.88) \\
PFOS & 12.30 (7.20-20.48) \\
PFHxS & 1.60 (0.90-2.80) \\
PFNA & 1.10 (0.80-1.64) \\
\bottomrule
\end{tabular}
\end{table}
\end{frame}

% Slide 7: Main Results - Forest Plot
\begin{frame}{Main Results: UNEXPECTED Inverse Associations}
\begin{center}
\includegraphics[width=0.9\textwidth]{../04-analysis/outputs/figures/figure4_forest_plot.png}
\end{center}
\vspace{-0.3cm}
\footnotesize
\textbf{All four PFAS}: Significant \emph{inverse} associations with PhenoAge acceleration\\
Higher PFAS $\rightarrow$ LOWER biological aging (opposite of hypothesis!)
\end{frame}

% Slide 8: Dose-Response Curves
\begin{frame}{Dose-Response Relationships}
\begin{center}
\includegraphics[width=0.9\textwidth]{../04-analysis/outputs/figures/figure5_dose_response.png}
\end{center}
\vspace{-0.3cm}
\footnotesize
Monotonic inverse dose-response relationships across the exposure range
\end{frame}

% Slide 9: Discussion - Why Paradoxical?
\begin{frame}{Discussion: Explaining Paradoxical Findings}
\textbf{Why inverse associations?} (Opposite of biological plausibility)

\vspace{0.3cm}

\textbf{Likely Explanations:}
\begin{enumerate}
    \item \textbf{Survival Bias}: Sicker individuals with high PFAS may die before enrollment
    \item \textbf{Reverse Causation}: Better kidney function $\rightarrow$ higher PFAS retention
    \item \textbf{Unmeasured Confounding}: SES, lifestyle, healthcare access
    \item \textbf{Sample Selection Bias}: Complete-case analysis selects healthier participants
    \item \textbf{Cross-sectional Limitation}: Cannot establish temporality
\end{enumerate}

\vspace{0.3cm}

\textbf{Comparison to Yan et al. 2025:}
\begin{itemize}
    \item Longitudinal study found \emph{positive} associations
    \item Reinforces importance of study design
\end{itemize}
\end{frame}

% Slide 10: Conclusions and Future Directions
\begin{frame}{Conclusions and Future Directions}
\textbf{Key Findings:}
\begin{itemize}
    \item All four PFAS showed significant \emph{inverse} associations with PhenoAge acceleration
    \item Results contradict biological plausibility and longitudinal evidence
    \item \textbf{Cross-sectional designs are insufficient for causal inference}
\end{itemize}

\vspace{0.3cm}

\textbf{Implications:}
\begin{itemize}
    \item \textbf{DO NOT} interpret as protective effects of PFAS
    \item Highlights methodological pitfalls in aging biomarker research
    \item Need for careful consideration of survival and selection biases
\end{itemize}

\vspace{0.3cm}

\textbf{Future Research:}
\begin{itemize}
    \item Longitudinal studies with repeated PFAS and aging measurements
    \item Multi-biomarker approaches (epigenetic clocks, telomere length)
    \item Life-course perspective: early-life exposure effects
    \item Investigate non-linear relationships and exposure mixtures
\end{itemize}

\vspace{0.3cm}

\centering
\textbf{Thank you!} \\ \texttt{elwoodresearch@gmail.com}
\end{frame}

\end{document}
